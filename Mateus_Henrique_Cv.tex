\documentclass[11pt,a4paper,sans]{moderncv}
\moderncvstyle{classic}

\usepackage[scale=0.80]{geometry}

\makeatletter

\moderncvcolor{blue}

\nopagenumbers{}

\firstname{Mateus}
\familyname{Henrique Ferreira da Silva}

\begin{document}

\lhead{Mateus Henrique Ferreira da Silva - Programador Júnior}
\renewcommand{\headrulewidth}{0.4pt}
\renewcommand{\footrulewidth}{0.4pt}

\section{Informações pessoais}

\cventry{Idade}{24 anos}{}{}{}{}
\cventry{Endereço}{Quadra 02 Torre E2 Apartamento 01}{Jardins Mangueiral}{São Sebastião}{Distrito Federal}{}
\cventry{Telefone}{(61)98658-7467}{}{}{}{}
\cventry{Email}{mateusrose7@gmail.com}{}{}{}{}
\cventry{Github}{github.com/mhenriquedev}{}{}{}{}


\section{Educação}

\cventry{Jun. 2015  -- Jul. 2019}{Sistemas de informação}{UDF}{}{}{}


\section{Experiência}
\cventry{Mar, 2018 - Fev, 2019}{Stefanini - Fábrica de Software - Cliente: Banco do Brasil}{}{}{}{
\begin{itemize}
\item Visitas e Fiscalização - solução web e mobile para auxiliar os gerentes do banco a fiscalizar empréstimos e agendar visitas a clientes para analisar a situação das garantias, se o empréstimo está sendo aplicado corretamente, etc. Foi criado um WebService utilizando a API JAX-RS (Jersey), CDI, Hibernate, Maven. Este serviço foi consumido tanto na Web como no Mobile. As aplicações possuem funcionalidades como Offline, integração com hardware, binários (download e upload de arquivos), geração de relatórios, geração de formulários dinâmicos, etc.
\linebreak
\item Desenvolvimento web: 
  \begin{itemize}
  \item Criação e manutenção de telas utilizando AngularJS
  \item Criação e manutenção de API Restful em Java
  \end{itemize}
\item Desenvolvimento mobile: 
  \begin{itemize}
  \item Criação e manutenção de telas utilizando React Native com TypeScript
  \item Criação e manutenção de API Restful em Java
  \end{itemize}
\end{itemize}}


\cventry{Fev, 2019 - Atual}{Trix Ti - Fábrica de Software}{}{}{}{
\begin{itemize}
\item Sistema de Atendimento Web (SAW) - solução web desenvolvida em Struts 2, arquitetura MVC, é uma ferramenta que integra todos os envolvidos no processo de atendimento a beneficiários de plano de saúde de operadoras. É um sistema robusto e com diversas funcionalidades. Atuação ocorreu em correção de bugs e criação de novas funcionalidades, bem como evolução de alguns módulos do sistema.
\linebreak
\item Desenvolvimento web: 
  \begin{itemize}
  \item Criação e manutenção de sistema legado em Struts 2 (JSP e Java)
  \end{itemize}
\end{itemize}}

\cventry{}{}{}{}{}{
\begin{itemize}
\item Novo Sistema de Atendimento (NSA) - solução web desenvolvida utilizando as tecnologias JSF 2.2, EJB 3.2, JPA, CDI. Surgiu para diminuir a complexidade do SAW, atendendo a todos os novos casos de uso e servindo como um "módulo" do SAW. Além das tecnologias citadas, utiliza PrimeFaces e jQuery.
\linebreak
\item Desenvolvimento web: 
  \begin{itemize}
  \item Criação de casos de uso em JSF.
  \end{itemize}
\end{itemize}}

\section{Idiomas}

\cvitem{Português}{Fluente }
\cvitem{Espanhol}{Básico }
\cvitem{Inglês}{Avançado }

\section{Habilidades}

\cvitem{Programação}{Java, C++, JavaScript, TypeScript}
\cvitem{SO}{Windows, Linux}
\cvitem{Frameworks}{Spring MVC, Springboot, JSF, Struts 2, Jasper, Jersey(JAX-RS), Struts, Hibernate, Collections, AngularJS, Angular 7}
\cvitem{Versionamento}{Git, SVN}
\cvitem{Servidores}{Tomcat, Wildfly, Glassfish}
\cvitem{Database}{MySQL, Oracle}
\cvitem{Testes}{JUnit}
\cvitem{Metodologias}{Scrum, Kanban}

\section{Cursos}

\cventry{Jan. 2018}{Java para Web com Hibernate, Servlet, JSP, JPA, JEE, JDBC, MVC, Struts}{Caelum}{40hs}{}{}
\cventry{Jan. 2018}{Java com Spring e testes - Spring, Git, Web Services, Design Pattern, Maven}{Caelum}{40hs}{}{}
\cventry{Fev. 2018}{Workshop de programação Java EE 7 - JSF, PrimeFaces, CDI , JPA com Hibernate, EJB}{Stefanini}{20hs}{}{}
\cventry{Abr. 2018}{Workshop de ReactJS, React Native com TypeScript}{Stefanini}{12hs}{}{}

\end{document}